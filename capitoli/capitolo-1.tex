% !TEX encoding = UTF-8
% !TEX TS-program = pdflatex
% !TEX root = ../tesi.tex

%**************************************************************
\chapter{Introduction}
\label{cap:introduction}
%**************************************************************
The exponential technological development and the internet have proliferated new criminally relevant conduct: new and more powerful IT tools have created new ways of committing a crime and have generated criminal phenomena.
\\For this reason, the area of primordial interest was the cybercriminal reality linked to the growing and hyperbolic development of the web and the technologies connected to it and how the types of crime (and criminals) and the most well-known and ordinary modus operandi have changed their physiognomy and have evolved with it.
\\Just think of online social networks and how they have an increasing impact on everyday life. Nowadays, social networks occupy a considerable part of daily life: they allow you to share links, photos, videos, thoughts, and to show your interest in some brands or companies. Many companies keep their customers updated through these channels, which can even be very useful for the growth of their customers, with shrewd and studied marketing skills.
In a more circumstantial reality, social networks allow you to stay in touch with friends near and far and favor the birth and development of relationships even with unknown users.
\\It's precisely in this context that the project takes shape. 
\\To start a relationship between two users, they must make contact, in some ways: generally or with a follow action (without some permissions) or with a friend request, where the user that receives the friend request has to accept (or not) the request to make contact with the other user. 
\\In the case of the social network par excellence, namely Facebook, a profile A asks for friendship to a profile B, who can choose whether to accept, leave pending or reject.
\\In fact, the person behind a virtual profile could be unknown: a person may have created a profile that does not belong to any real person but simulating its existence, to get in touch with some specific users.
\\This mechanism has led to the birth of online crime aimed at targeting specific people who, initially unaware, find themselves becoming \textit{victims}.\\From here, a first project idea was born: put the focus from the attacker's point of view and then create a tool to analyze a victim and suggest to the criminal a series of information be used to create a profile that the victim would more likely accept.

%\noindent Esempio di utilizzo di un termine nel glossario \\
%\gls{api}. \\

%\noindent Esempio di citazione in linea \\
%\cite{site:agile-manifesto}. \\

%\noindent Esempio di citazione nel pie' di pagina \\
%citazione\footcite{womak:lean-thinking} \\

%**************************************************************
\section{Organizzazione del testo}

\subsection{Contents}

\begin{description}
    \item[{\hyperref[cap:processi-metodologie]{Il secondo capitolo}}] descrive ...
    
    \item[{\hyperref[cap:descrizione-stage]{Il terzo capitolo}}] approfondisce ...
    
    \item[{\hyperref[cap:analisi-requisiti]{Il quarto capitolo}}] approfondisce ...
    
    \item[{\hyperref[cap:progettazione-codifica]{Il quinto capitolo}}] approfondisce ...
    
    \item[{\hyperref[cap:verifica-validazione]{Il sesto capitolo}}] approfondisce ...
    
    \item[{\hyperref[cap:conclusioni]{Nel settimo capitolo}}] descrive ...
\end{description}

\subsection{Typographic conventions}
Riguardo la stesura del testo, relativamente al documento sono state adottate le seguenti convenzioni tipografiche:
\begin{itemize}
	\item gli acronimi, le abbreviazioni e i termini ambigui o di uso non comune menzionati vengono definiti nel glossario, situato alla fine del presente documento;
	\item per la prima occorrenza dei termini riportati nel glossario viene utilizzata la seguente nomenclatura: \emph{parola}\glsfirstoccur;
	\item i termini in lingua straniera o facenti parti del gergo tecnico sono evidenziati con il carattere \emph{corsivo}.
\end{itemize}