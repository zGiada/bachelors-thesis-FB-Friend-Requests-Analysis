% !TEX encoding = UTF-8
% !TEX TS-program = pdflatex
% !TEX root = ../tesi.tex

%**************************************************************
\chapter{Conclusions}
\label{cap:conclusions}
\section{Summary}
The ``\texttt{Zero-Effort Attack}'' tool, which was the objective of this internship, has been developed and is fully functional. So the goal was achieved brilliantly and much food for thought was reported for any future in-depth work.
%**************************************************************
\section{Acquired knowledge}
During this internship, the student had the opportunity to deepen and put his knowledge into practice, in particular:
\begin{itemize}
	\item in the field of the automated management of browsers through the use of the Selenium framework;
	\item Python programming;
	\item tools and methodologies for data collection.
\end{itemize}
%**************************************************************
\section{Personal evaluation}
This project allowed me to deepen and put into practice knowledge related to programming with Python and Selenium and, above all, it allowed me to get my hands in one of the fields of Cybersecurity, which is the safety of people in a now ordinary place like social networks.\\
It would have been much more interesting to have been able to develop even just one of the points presented in Chapter 9, but the short time and optimal planning that was nevertheless subject to changes to solve problems of an unpredictable nature didn't make possible the more in-depth development of the algorithm.