% !TEX encoding = UTF-8
% !TEX TS-program = pdflatex
% !TEX root = ../tesi.tex

%**************************************************************
\chapter{Future works}
\label{cap:future-works}
%**************************************************************
As also specified at the beginning of chapter \ref{cap:data-analysis}, the analysis and assessments reported are based on the results that have emerged so far with the data collected. \\ For example, it could be considered that many profiles who have not accepted friendship could be simply inactive, since for many now Facebook is no longer the reference social network) or not very present in social networks.
A list of possible future works is therefore now made, starting from these considerations.

\section{More than 3 friend requests}
Going back to the discussion made in chapter \ref{cap:data-analysis}, the number of friend requests that each attacking profile sends could be greater and this would lead to greater accuracy of the results found. \\The tools that manage the organization of friend requests (chapter \ref{cap:tool-friend-request}) have been built to manage a very large amount of data, so already with this code created, you could deepen the request by increasing the number of requests, to see if and how the final result changes.

\section{"Current town", "occupation" and younger users}
It should be noted that other parameters could be interesting to study, which are "working occupation" and "current town". \\
In particular, often the friend request is dictated by work/study interests, so checking the job occupation that the profile declares to employ. \\
Same for the current town: maybe a profile is more secure to accept friend requests only for those profiles that indicate a current town like its o near its, or is this indifferent?\\
With a case study that also includes these parameters, the accuracy would certainly be greater and perhaps it would open the way to probably more interesting scenarios, although perhaps more linked to a psychological field.\\
In the same way, it could be considered among the various age ranges, even a range for minors, who are very often the weakest users and the victims preferred by criminals.\\The problem is that there are not many profiles that claim to be minors because Facebook is no longer the social network used by these new generations; moreover, those few profiles that exist almost never make their age visible even to non-friends.\\
Considering that Facebook does not allow searches for people based on age, it is very difficult to be able to include minors in some case studies, even if it would be very interesting and useful.

\section{Profile pictures}
As already reported in chapter \ref{cap:alt-technology} and demonstrated in chapter \ref{cap:discuss-image-profile}, profile pictures greatly influence the outcome of a friend request. The results show that if the attacking profile has a real image it is more easily accepted by the victims, and at the same time, a victim profile with the real profile image is more likely to accept friend requests than a profile with a hidden image.\\
An example of future work could be to create multiple profiles with different ages belonging to the same range (or by dividing the age into a larger number range) in order to effectively verify how much the profile image of a real person can influence the analysis, and to cross-reference the data in cases where the profile picture is false.
\section{More accurate detection of profile pictures}
A future job could certainly be to develop a more accurate detection system of profile images. As an example, \textbf{YOLO} (\textbf{Real-Time Object Detection}, detailed here: \url{https://pjreddie.com/darknet/yolo/}) is reported.\\
YOLO is a state-of-the-art, terminal-based, real-time object detection system that analyzes an image and declares which elements are part of it and with what percentage. An image of a person turned from behind was not always classified as a true image, as Facebook's technology could not recognize the person present and therefore did not insert in the \texttt{alt} tag.\\
Among other things, some recent articles report how the Facebook technology that determines the automatic content of image \texttt{alt} tags, is continuously improving for the detection of more details.\\
In this case study, due to timing issues, it was not possible to completely fine-tune the use of YOLO, but we want to underline however how the percentage of error that the current system produces is very low (about 2 profiles every 50 examined), and that to remedy any possible inaccuracies, a check system has been implemented, discussed in chapter \ref{cap:classificator-profiles}.

\section{Check data}
A big problem in this case study is checking that the data a user has entered is valid, correct, and consistent.
\\For example, a 30-year-old man can safely write that as a job he is a Milan footballer when it is not blatantly the true. \\Similarly, a user could write unclear things, for example, more than one profile was found that reported "I work for myself" as a job. And it might well be right, but categorizable in no way. \\In the end, many people write that I work in a particular place, but they do not specify their role: for example, a mechanic might write the name of the shop where he works and not the fact that his job is to be a mechanic.
\\So a future work could be to take into consideration all the possible variant and to know how to evaluate and analyze them.
