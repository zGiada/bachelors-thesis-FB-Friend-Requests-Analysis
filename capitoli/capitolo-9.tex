% !TEX encoding = UTF-8
% !TEX TS-program = pdflatex
% !TEX root = ../tesi.tex

%**************************************************************
\chapter{Future works}
\label{cap:future-works}
%**************************************************************
I dati riportati sono basati sui risultati emersi fino ad ora.\\(Si potrebbe considerare che molti profili potrebbero essere inattivi oppure poco presenti nei social). \\Da considerare per sviluppi futuri:
\section{More than 3 friend requests}
%i tool sono costruiti per gestire anche numeri maggiori

\section{"Current town" and "occupation"}
%After this specification, it should be noted that other parameters could be interesting to study, which are "working occupation" and "current town". \\
%In particular, often the friend request is dictated by work/study interests, so checking the job occupation that the profile declares to employ. \\
%Same for the current town: maybe a profile is more secure to accept friend requests only for those profiles that indicate a current town like its o near its, or is this indifferent?\\
%The problem is that it just takes a look at user profiles that specify these other parameters to realize that many people enter fake or fictional values and therefore the result of the analysis considering also these values can hardly be considered valid.

\section{Minorenni}

\section{YOLO}
%Una versione più avanzata di questo algoritmo sfrutterà YOLO (Real-Time Object Detection, qui nel dettaglio: \url{https://pjreddie.com/darknet/yolo/}). Per questioni di tempistiche non è stato possibile mettere a punto in maniera completa l'utilizzo di questa tecnologia. \\Si vuole sottolineare comunque come la percentuale d'errore che il sistema attuale produce è davvero molto bassa (circa 2 profili ogni 50 esaminati), e che per ovviare a qualche possibile imprecisione, è stato implementato un sistema di \textit{check}, della quale si parlerà in seguito.

\section{Check data}
%First of all, it should be noted that no additional checks on data consistency have been made, as it is assumed that the people who post their information on Facebook are sincere and the data is complete. The only check system is on the categorization tool of the collected profile (see here).\\
