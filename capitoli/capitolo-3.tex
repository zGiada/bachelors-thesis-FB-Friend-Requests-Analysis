% !TEX encoding = UTF-8
% !TEX TS-program = pdflatex
% !TEX root = ../tesi.tex

%**************************************************************
% !TEX encoding = UTF-8
% !TEX TS-program = pdflatex
% !TEX root = ../tesi.tex

%**************************************************************
\chapter{Background}
\label{cap:background}
This chapter describes the main technologies used in this project: Facebook, Python, and Selenium.
%**************************************************************
\section{Web Scraping}
For this project, the information necessary, that permits the collection of analyzable data, is situated on the Facebook pages: this information must be found and save. To do it, the Facebook pages need to be scraped, which means exploit the techniques and the technologies to extract data from a website. But, scrape webpages manually every single day is a waste of time: a lot of time will be spent to click, scroll and search the desire information. Furthermore, the content of these pages could be updated very frequently, thus scraping should be periodic. For this reason, in this project automated web scraping will be ordinary. The information will be collected every single day in order to have a big quantity of data to analyze in a second moment, but it will happen in an automated way: the code will be written only once, but it will be used for every needed page. Thus, appropriate tools will be developed: they will scrape the web pages that contains the data to collect and they will save them in the specific files. But never forget that the data on the websites are unstructured: a tool that scrapes a web page must understand what kind of information have to scrape and collect, and then it must save them in a structured way, depending on the context.

\section{Python language and Selenium}
\textbf{Python}\parencite{site:python} is an interpreted high-level general-purpose programming language. It is easy to read, thanks to its use of significant indentation. It is easy to learn, simple to code, with a clear syntax and easier maintenance. It supports modules and packages, which encourages program modularity and code reuse. Furthermore, a small code is enough to do large tasks. Thus, Python is suitable for web scraping because of all these reasons and, above all, because it has a huge collection of libraries that provides methods and services for various purposes, among which, libraries for manipulation of extracted data. One of these libraries is called Selenium.
\par \noindent \textbf{Selenium}\parencite{site:selenium} is an open-source web testing library, used to automate browser activities: a browser-driver executes a script on a browser instance on a device, and this permits to automate and test web algorithms. In this project, Selenium will be used to automate procedures that scrape Facebook pages in order to collect data from them. More specifically, it will be used Selenium WebDriver. \textbf{Selenium WebDriver}\parencite{site:webdriver} is a web framework used for automating web-based application testing to verify that it performs expectedly: it simulates the behavior of a real user within a browser.	

\section{Facebook}
\textbf{Facebook}\parencite{site:facebook} is a social network, born in February 2004. 
It permits people to share content: Facebook users could create a personal profile, add other users as friends, and exchange messages. Additionally, users could share their status, news stories, notes, photos, videos, and allow their friends (or friends of friends) to comment on them. Furthermore, users may join common-interest groups, organize events, and create fans pages for a workplace/business, a school/college, or even a brand/product. \par \noindent
Right now, Facebook is not as popular as it used to be, especially among the younger ones. Younger generations prefer to use others social networks, like Instagram or Tik Tok: in fact, in recent times Facebook has seen the average age of its users increase \parencite{site:gen-z}. But, despite this, it is currently still the most used social network: Facebook has over 2.7 billion monthly active users, ahead of Instagram with its 1.2 billion monthly active users and other social networks such as Twitter or TikTok. \parencite{site:top-social}
\par \noindent Furthermore, Facebook is not a new area in literature, but compared to other OSNs like Instagram or Twitter, has many more analyzable factors, so there is a lot more material to study and then critically examine the resulting data of this project.\par 	\noindent
In short, Facebook makes the world and the people more open and connected.
However, it is unavoidable that this platform may also provide incentives for criminals to carry out illegal activities, as explained in the next section.

\subsection{Spotlight on security in social networks}
This project was born after reflecting on security in social networks. In an increasingly interconnected world, everything becomes virtual more and more rapidly, including crimes involving people themselves in the flesh, not just the digital identity of a person, that is the data that the person put online.\par \noindent
This project wants to shed light not on the technological defects that may exist, but on the people who are inside the Internet, behind a virtual profile.\par 	\noindent
Facebook uses a policy system to try to prevent problems related to the safety of people on the platform: starting from the constraint of minimum age to register, ending to checks for criminal or inauthentic behavior. Fake profiles, in fact, have always been a problem for all platforms that allow users to register. A person can create a profile and be real to themselves, but at the same time, can create a fake profile that identifies another person (existent or not). \par 	\noindent
This thesis wants to highlight that behind a virtual profile you can never be sure of who really is and show the weaknesses of the average user who uses these platforms.