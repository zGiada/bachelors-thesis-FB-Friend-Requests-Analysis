% !TEX encoding = UTF-8
% !TEX TS-program = pdflatex
% !TEX root = ../tesi.tex

%**************************************************************
% Sommario
%**************************************************************
\cleardoublepage
\phantomsection
\pdfbookmark{Sommario}{Sommario}
\begingroup
\let\clearpage\relax
\let\cleardoublepage\relax
\let\cleardoublepage\relax

\chapter*{Abstract}

This document describes the project developed during the internship period. The internship, which lasted about 320 hours, was carried out in smart working and in collaboration with the SPRITZ Security and Privacy Research Group.\par \noindent 
The topic is about studying the phenomenon of friend requests on Facebook and how an attacker profile with certain characteristics is more easily accepted by a victim profile with certain characteristics. In the first place, the study of the environment was required to outline the requirements and objectives. Then it was required the implementation of tools that allow the creation of attacker profiles, the search for profiles with certain characteristics, and the sending of the friend request. The collected data, then, had to be analyzed in order to create a \textit{model of acceptance}: it describes the acceptance percentage that a profile with determined characteristics has towards another profile with determined characteristics.\par \noindent 
This is because the final goal was the development of a tool called ``\texttt{Zero-Effort Attack}'', which, given a victim profile, defines the characteristics that the attacker profile should assume to be accepted more easily.

%\vfill
%
%\selectlanguage{english}
%\pdfbookmark{Abstract}{Abstract}
%\chapter*{Abstract}
%
%\selectlanguage{italian}

\endgroup			

\vfill

